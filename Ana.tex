% !TEX encoding = UTF-8 Unicode

\documentclass[a4paper]{article}

\usepackage{color}
\usepackage{url}
\usepackage[T2A]{fontenc} % enable Cyrillic fonts
\usepackage[utf8]{inputenc} % make weird characters work
\usepackage{graphicx}

\usepackage[english,serbian]{babel}
%\usepackage[english,serbianc]{babel} %ukljuciti babel sa ovim opcijama, umesto gornjim, ukoliko se koristi cirilica

\usepackage[unicode]{hyperref}
\hypersetup{colorlinks,citecolor=green,filecolor=green,linkcolor=blue,urlcolor=blue}

%\newtheorem{primer}{Пример}[section] %ćirilični primer
\newtheorem{primer}{Primer}[section]

\begin{document}

\title{Kvantna kriptografija\\ \small{Seminarski rad u okviru kursa\\Tehničko i naučno pisanje\\ Matematički fakultet}}

\author{Igor Glišović\\mi22292@alas.matf.bg.ac.rs\\\\ Željko Zekavičić\\mi22130@alas.matf.bg.ac.rs\\\\ Nađa Lazarević\\mi22175@alas.matf.bg.ac.rs\\\\ Ana Mladenović\\mi22119@alas.matf.bg.ac.rs\\\\}
\date{24.~oktobar 2017.}
\maketitle

\abstract{

\newpage

\tableofcontents

\newpage

\section{Uvod}
\label{sec:uvod}
Kriptografija je nauka koja se bavi očuvanjem tajnosti informacija. Cilj je da se informacije prenesu od pošiljaoca do primaoca tako da smo sigurni da one nisu dospele do nekog trećeg lica. Pri samom nastanku kriptografije glavni problemi su bili na koji način sačuvati tajnost procesa enkripcije/dekripcije podataka (neki od prvih primera klasične kriptografije su Cezarova šifra\footnote{Cezarova šifra (1. vek p.n.e) se zasnivala na supstituciji, gde bi se pri šifriranju svaki karakter pomerio unapred za n-pozicija u alfabetu, a pri dešifrovanju pomerio unazad za n-pozicija}  i Skital\footnote{Skital (5. vek p.n.e) je parče štafete koji su Grci koristili kako bi obavijali sakrivenu poruku oko nje i nakon toga došli do ispravne poruke}), jer bi se nakon saznavanja procesa na koji se šifruju podaci, oni mogli lako dešifrovati.

Nakon mnogo vekova, nastankom i razvojem informacionih tehnologija, očuvanje sigurnosti podataka postaje još bitnije (ali i teže) nego što je nekad bilo. Kao jedan od najboljih načina za enkripciju podataka iz ere klasične kriptografije izdvaja se Vernamova šifra (tzv. \textit{OneTimePad} - OTP). OTP se zasniva na principu generisanju nasumičnog dugačkog ključa dužine kao izvorna poruka koju šaljemo. Ovakav sistem radi samo ako svaki put generišemo novi kod, jer ako dva puta prosledimo isti, lako se može dešifrovati ono što smo poslali. Takođe, veliki problem kod ovakvog načina enkripcije predstavlja prenos dugačkog ključa, te bi sam proces slanja poruke trajao duže. Ovaj problem se rešava tako što ključ skraćuje na fiksnu dužinu za svaku poruku. Međutim, moderna kriptografija sada nailazi na novi problem (poznatiji kao Kvaka 22):

 \begin{center}
\textit{Komunikacija između pošiljaoca i primaoca je sigurna, samo ako znamo da je sigurna.}
\end{center} 

Ovde dolazimo do zaključka da se klasičnom kriptografijom nikako ne može znati da li je informacija kompromitovana od strane trećih lica u toku slanja poruke. Ovaj problem se rešava primenom osobina kvantne mehanike na prenos informacija u nekom informacionom sistemu.

\section{Principi kvantne kriptografije}
.
\subsection{Kvantni računari}
.
\subsection{Kvantni protokoli}
.
\subsection{Kvantni protokoli}
.
\section{Istorijat}	
\label{sec:termini_i_citiranje}

.
\section{Kvantna kriptografija danas}
Koristeći BB84 protokol sa (decoy) lažnim pulsevima Univerzitet u Kembridžu u saradnji sa kompanijom Toshiba postiže sistem koji razmenjuje sigurne ključeve na brzini od 1 Mbit/s (preko 20 km optičkih vlakana) i 10 kbit/s (preko 100 km vlakana). On danas ima najveću brzinu prenosa. 
2007. godine počevši od marta najveća udaljenost na kojoj je razmena kvantnog ključa demonstrirana koristeći optička vlakna je 148.7km, ostvarena od strane Los Alamos National Laboratory/NIST grupe koristeći BB84 protokol. Bitno je to da je distanca dovoljno velika za svako prožimanje koje može biti potrebno u današnjim mrežama od vlakana. Između dva Kanarska ostrva ostvareno je najveće rastojanje za DKK u slobodnom prostoru koje iznosi 144km. Postignuto je od strane evropskog udruženja pomoću isprepletene fotone (Ekertova šema) 2006. godine i koristeći modifikovan BB84 protokola u 2007. Zahvaljujući nižoj gustini atmosfere na većim visinama, prenos do satelita moguć.
 Švajcarska kompaija Id Quantique poseduje tehnologiju za kvantnu enkripciju koja se koristila u ženevskom kantonu za prenos izbornih rezultata do
prestonice 21. oktobra 2007.
Prvi bankovni transfer koji je koristio kvantnu kriptografiju bio je u Beču 2004. godine. Od gradonačelnika ovog grada do Austrijske banke prenesen je važan ček za koji je bila potrebna apsolutna sigurnost. Prva svetska
računarska mreža zaštićena kvantnom kriptografijom je implementirana u oktobru
2008. na naučnoj konferenciji u Beču. Ime mreže je SECOQC (Secure
Communication Based on Quantum Cryptography), a Evropska unija je finansirala
projekat. Kako bi međusobno povezala šest lokacija u Beču i mestu Sant Polten, koji se nalazi 69 km
zapadno od Beča mreža koristi standardni kabl od optičkih vlakana dugačak 200 km.\\
Kvantni računari se danas još više razvijaju, a komercijalne kvantne kriptografske sisteme danas mogu da ponude čak četiri kompanije: id Quantique (Ženeva), MagiQ Technologies (Njujork), SmartQuantum (Francuska) i Quintessence Labs (Australija). U ovoj oblasti postoji nekoliko kompanija koje poseduju aktivne istraživačke programe uključujući kompanije Toshiba, HP, IBM, Mitsubishi, NEC i NTT. 
\section{Zaključak}
\label{sec:zakljucak}

.

\newpage
\addcontentsline{toc}{section}{Literatura}
\appendix

\iffalse
\bibliography{seminarski} 
\bibliographystyle{plain}
\fi

\begin{thebibliography}{9}

\bibitem{laski2009software} J. Laski and W. Stanley. \emph{Software Verification and Analysis}. Springer- Verlag, London, 2009.

\bibitem{gcc} Free Software Foundation. GNU gcc, 2013. on-line at: http://gcc. gnu.org/.

\bibitem{haltingproblem} A. M. Turing. \emph{On Computable Numbers, with an application to the Entscheidungsproblem}. Proceedings of the London Mathematical Society, 2(42):230–265, 1936.


\end{thebibliography}


\appendix

\section{Dodatak}
Ovde pišem dodatne stvari, ukoliko za time ima potrebe.
Ovde pišem dodatne stvari, ukoliko za time ima potrebe.
Ovde pišem dodatne stvari, ukoliko za time ima potrebe.
Ovde pišem dodatne stvari, ukoliko za time ima potrebe.
Ovde pišem dodatne stvari, ukoliko za time ima potrebe.


\end{document}
